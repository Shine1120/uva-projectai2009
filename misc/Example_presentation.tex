\documentclass{beamer}
\usepackage{beamerthemesplit}
\usetheme{classic}%\usetheme{Boadilla} %\usetheme{shadow} 
\title{Language and simulation in conceptual processing}
\author{Lawrence W Barsalou, Ava Santos, \\W Kyle Simmons, Christine D Wilson}
\date{}

\begin{document}
\frame{\titlepage}
\section[Outline]{}
\frame{\tableofcontents}
\definecolor{orange}{rgb}{0.9,0.4,0}

\section{LASS theory}
\subsection{Linguistic motivated representations of knowledge}
\frame{
  \frametitle{Knowledge representation theories}
	\hspace*{8px}The authors focus on 2 sources of knowledge representation: 
  \begin{itemize}
  \item \textbf{linguistic forms} in the brain\rq s language systems 
  \item \textbf{situates simulations} in the brain\rq s modal system (perceptual system)
  \end{itemize}
}
\frame{
  \frametitle{Knowledge representation theories}
	\hspace*{8px}Some theories argue that there is no underlying system of amodal symbols that correspond to language. They state that there are just linguistic form and \textbf{the knowledge is just the statistical distribution} of these forms.\\
	\vspace{10px}\pause
	\hspace*{8px}Some older theories assert that knowledge is grounded in the brain\rq s amodal system which captures modal states during perception, action and introspection and, later, simulates these states to represent knowledge (knowledge about something is \textbf{simulated in the context of a likely background situation}).    
}
\subsection{Language and Situated Simulation(LASS) Theory}
\frame{
  \frametitle{LASS theory of conceptual processing}
	\hspace*{8px}The authors start with the idea that humans are focusing attention on object, actions, events, mental states in order to produce concepts of certain categories.\\
	\hspace*{8px}The \emph{Language and situated simulation (LASS)} framework proposed by the authors is represented by:
  	\begin{itemize}
		\item Linguistic Processing
		\item Situated Simulation
		\item Mixture of interactions of language and situated simulations
		\item The statistical underpinning of language and situated simulations 
	\end{itemize}
}
\frame{
  \frametitle{1. Linguistic processing}
	\hspace*{8px}The authors assert that both systems are active: the \emph{language system} and the \emph{situated simulation system}. But when a word is perceived the linguistic system is activated first and the situated simulation system second.\\
%	\begin{figure}[!hbtp]
%	\centering
%	\includegraphics[width=0.6\textwidth]{lass.jpg}
%	\end{figure}
}
\frame{
  \frametitle{1. Linguistic processing}
	\hspace*{8px}The fact that the language system peaks first is explained by the fact that one word elicits other words by simple \textbf{word association}.\\
	\vspace{15px}\pause
	\hspace*{8px}There are many findings consistent with the proposal that \textbf{linguistic forms can be processed superficially.}\\
	\vspace{15px}\pause
	\hspace*{8px}The authors of the article \textbf{do not believe that the language system contains any semantic content that could produce deep conceptual understanding}.
}

\frame{
  \frametitle{2. Situated Simulation}
	\hspace*{8px}After the language system has recognized the presented word, the word immediately begins to activate associated simulations, preparing the agent for situated action.\\
	\vspace{10px}\pause
	\hspace*{8px}The simulations contain conceptual content about properties and relations.\\
	\vspace{10px}\pause
	\hspace*{8px}The simulations \textbf{provide the meaning of the linguistic forms}. The authors state that using only linguistic forms would be like manipulating symbols in an unfamiliar language with no true comprehension.\\
	\vspace{10px}\pause
	\hspace*{8px}Simulations are roughly equivalent to concepts in the traditional theories of knowledge.
}
\frame{
  \frametitle{3. Mixtures and interactions of language\\ and situated simulations}
	\hspace*{8px}There are extensive interactions between these 2 systems.\\
	\vspace{15px}\pause
	\hspace*{8px}People engage simultaneously in simulating the relevant situation and verbalizing about it.\\
	\vspace{15px}\pause
	\hspace*{8px}Simulation represent the \textbf{content of thought}, while words provide \textbf{tools for indexing and manipulating this content}. 
}
\frame{
  \frametitle{4. Statistical underpinning of language\\ and situated simulations}
	\hspace*{8px}Simulations capture the \textbf{statistical frequencies of properties} and the \textbf{relations} between them in experience.\\
	\vspace{15px}\pause
	\hspace*{8px}The \textbf{frequencies of words, the associations between} them and the \textbf{relations to syntactic structures} are also coded statistically.\\
	\vspace{15px}\pause
	\hspace*{8px}\textcolor{orange}{In conclusion frequencies and correlations in perceived situations are mirrored in frequencies and correlations in words we use to describe them}. 
}
\section{Previous theories}
\subsection{Paivio's dual code theory} 
\frame{
  \frametitle{Paivio's dual code theory}
	\hspace*{8px}Both are based on the idea that there are 2 basic systems: one that processes linguistic representation and another that processes modal representations.\\
	\hspace*{8px}Unlike the LASS, the dual code theory assumes that the linguistic system is central.\\
	\hspace*{8px}Evidence for the existence of 2 systems can be found in literature related to:
	\begin{itemize}
	\item developmental psychology  
	\item individual differences
	\item episodic memory and semantic memory
	\item language comprehension
	\end{itemize}
}
\subsection{Glaser\rq s lexical hypothesis} 
\frame{
  \frametitle{Glaser\rq s lexical hypothesis}
	\hspace*{8px}Glaser showed that pictures are faster than words in accessing the conceptual system. While pictures access the system directly, the words access the linguistic system first.\\
	\hspace*{8px}Similarities between the theories:
	\begin{itemize}
	\item Both theories (LASS and Glaser\rq s) assume the existence of 2 systems: a linguistic system and a conceptual system 
	\item Both assume that superficial linguistic processing can be sufficient under certain conditions
	\end{itemize}
	\hspace*{8px}The difference is that Glaser is less prone to view the conceptual system as containing modal representations.\\
}
\section{Experimental evidence}
\subsection{Experiment 1. Word association} 
\frame{
  \frametitle{Experiment 1. Word association}
	\hspace*{8px}Participants were asked to say the words that come into their minds immediately after hearing a certain word. The responses were grouped in the following 3 categories:
	\begin{enumerate}
	\item linguistically related responses: 	
			\begin{itemize}
				\item forward compound phrases (\emph{e.g. bee $\rightarrow$ hive}) 
				\item backward compound continuations (\emph{e.g. honey $\rightarrow$ bee})
				\item synonyms \slash antonyms
				\item sound similarities (\emph{e.g. bumpy $\rightarrow$ lumpy}) 
			\end{itemize}
	\item taxonomic responses
			\begin{itemize}
				\item super-ordinate categories (\emph{e.g. dog $\rightarrow$ animal})
				\item coordinate categories (\emph{e.g. dog $\rightarrow$ cat})
				\item subordinate categories (\emph{e.g. dog $\rightarrow$ terrier})
			\end{itemize}
	\item object-situation responses
		\begin{itemize}
			\item a property of the cue (\emph{e.g bird $\rightarrow$ wings})
			\item a thematic association of the cue concept (\emph{e.g. golf $\rightarrow$ sunshine})
		\end{itemize}
	\end{enumerate}
}
\frame{
  \frametitle{Experiment 1. Word association}
	\hspace*{8px}\textbf{Linguistic responses} tend to occur early in participants\rq protocols (generated by the linguistic system).\\
	\vspace{15px}
	\hspace*{8px}\textbf{Object-situation responses} occur relatively late (viewed as residing in the conceptual system).\\
	\vspace{15px}
	\hspace*{8px}\textbf{Taxonomic responses} are halfway between (some taxonomic responses are memorized in childhood such as: \emph{\lq dog - animal\rq}) and some others describe the content of simulations. 
}
\subsection{Experiment 2. Property generation} 
\frame{
  \frametitle{Experiment 2. Property generation}
	\hspace*{8px}Participants were asked to indicate what characteristics are typically true for a specific concept.\\
	\vspace{15px}
	\hspace*{8px}Participants produced \textbf{fewer linguistic responses} and more object-situation responses than in the word association experiment (the task had a more conceptual nature).\\
	\vspace{15px}
	\hspace*{8px}Super-ordinate taxonomies occurred earlier than the rest (\textbf{super-ordinates may be stored linguistically}).\\
	\vspace{15px}
	\hspace*{8px}The coordinate and subordinate were as slow as the object-situation responses. 
}
\subsection{Experiment 3. Property generation with fMRI} 
\frame{
  \frametitle{Experiment 3. Property generation with fMRI}
	\hspace*{8px}Experiment was performed using a 3-Tesla fMRI (functional Magnetic Resonance Imaging) scanner. The words for each concept were presented visually and the participants had to accomplish 3 tasks:
	\begin{itemize}
	\item indicate the properties of concepts
	\item create word associations
	\item imagine a situation that contained the concept
	\end{itemize}
}	
\frame{
  \frametitle{Experiment 3. Property generation with fMRI}
	\hspace*{8px}Word association task primarily activated the left-hemisphere language-areas.\\
	\vspace{15px}
	\hspace*{8px}The situation simulation task activated bilateral posterior areas that are typically involved in the generation of mental imagery.\\
	\vspace{15px}
	\hspace*{8px}For the property generation task activation was found in both localizer tasks. Activations found in the word association localizer occurred earlier than activations found in the situation localizer.  
}
\subsection{Experiment 4. Property verification} 
\frame{
	\frametitle{Experiment 4. Property verification}
	\hspace*{8px}Participants received an object name on a computer screen and subsequently a property had to be verified if it was part of the respective object.\\
	\vspace{15px}
	\hspace*{8px}When task conditions allowed, participants used a superficial \textbf{linguistic strategy}.\\
	\vspace{15px}
	\hspace*{8px}When deeper conceptual processing was required, participants \textbf{used simulations}.\\
}
\frame{
	\frametitle{Experiment 4. Property verification}
	\begin{itemize}
		\item \textbf{Linguistic strategy} - it is used when information in the linguistic task is sufficient for adequate performance. In this case the true properties of the concept given were related to the word, while the false property was unrelated (\emph{e.g. bathtub - drain}, \emph{e.g. bathtub $\rightarrow$ aircraft}).
		\item \textbf{Simulation strategy} - in this case the false properties were related to the word given (\emph{e.g. table $\rightarrow$ furniture}). The participants were not able to rely on the linguistic system and hat to retrieve conceptual information that specified whether the property is part of the object or not.
	\end{itemize}
}
\frame{
	\frametitle{Experiment 4. Conclusions}
	\hspace*{8px}Participants were over 100 milliseconds faster to verify the true trials when the false trials were unrelated, than when they were related.\\
	\hspace*{8px}Possible variables that could explain the variance would be:
	\begin{itemize}
	\item linguistic - the association strength between words 
	\item perceptual - the size of the properties or the salience of the properties
	\item expectation variables - variability of property forms
	\end{itemize}
	\hspace*{8px}The result showed that language and simulation systems were working in parallel. When a strong linguistic association was not found participants relied on the simulation system (the brain areas that process visual images were active, but when the false trials were unrelated this area was not active). 
}
\frame{
  \frametitle{Abstract concepts}
	\hspace*{8px}It was thought that concrete concepts are represented by the simulation system because the memory for concrete concepts is good, while abstract concepts are represented by the language system.\\
	\hspace*{8px}The authors of the article argue that this theory is not good because \textbf{people only understand concepts once they can ground the language in experience}.\\
	\hspace*{8px}Another experiment has been done in which participants were given an abstract concept and they had to verify whether the concept applied in a subsequent picture. The results showed that the linguistic system was less active for concrete concepts than for abstract concepts under these task conditions. 
}

\section{Relevance of LASS}
\subsection{Language comprehension} 
\frame{
	\frametitle{Language comprehension}
	\hspace*{8px}There are 2 types of memory:
		\begin{itemize}
		\item \textbf{surface memory} - reflects linguistic structure in the linguistic system
		\item \textbf{grist memory} - may reflect the simulation in simulation system
		\end{itemize}
	\hspace*{8px}The grist memory looses information about the specific form of the linguistic input, but the meaning of the sentence is integrated into a coherent semantic representation.\\
	\hspace*{8px}Another parallel with the 2 systems is inferencing:
		\begin{itemize}
		\item \textbf{minimal inferencing} - occurs when people process only the linguistic form and if they construct simulations, the may simulate the memory of individual words without integrating them into a coherent global system
		\item \textbf{rich inferencing}
		\end{itemize}
}
\subsection{Other examples} 
\frame{
	\frametitle{Other examples}
	\begin{itemize}
	\item Conceptual Processing - high frequency associations between words often produces semantic responses that describes events, while low frequency words are associated with the linguistic because there is not enough experience to activate familiar situations 
	\item Social Processes - there are also two systems present: the first one is based on fast associative information that is statistically likely (linguistic system); the second is characterized by more thoughtful processing that relies on careful reasoning about particular situation 
	\item Education - students may vary in how their understanding of a particular domain engages both language and simulation systems  
	\end{itemize}
}
\subsection{Complex and superficial linguistic processing} 
\frame{
	\frametitle{Complex and superficial linguistic processing}
	\hspace*{8px}The two systems are \textbf{highly dependent} and cycle between periods of activity and relative inactivity.\\ 
	\hspace*{8px}Deep conceptual processing may require the simulation system. For example in the property verification experiment, rather then storing the properties amodally as a relation: \emph{part(x,y)}, the problem was solved by simulating the property.\\
	\hspace*{8px}The authors assert that the simulation systems have existed long before humans and they were used to store memories that would later help to generate anticipatory inferences.\\
	\hspace*{8px}\textcolor{orange}{Linguistic systems contain considerable amounts of statistical information that mirrors the content of the simulation system, which in turn mirrors the content of experience.}
}
\subsection{The time course of processing simulations} 
\frame{
	\frametitle{The time course of processing simulations}
	\hspace*{8px}Why does executive processing selects the linguistic system first as a source of relevant information?\\
	\vspace{10px}\pause
	\hspace*{8px}The answer may be: because \textbf{the cues on all the tasks were words}. When linguistic systems is capable of generating a correct response there will be no need to go outside the system, when it cannot, it must shift to the simulation system.\\
	\vspace{10px}\pause
	\hspace*{8px}While \textbf{conceptual systems are probably heavily oriented towards processing nonlinguistic experience}, the \textbf{linguistic systems are used to draw attention towards important regions of experience} that are relevant for the situated action.  
}
%  \begin{itemize}
%  \item<1-> linguistic forms in the brain\rq s language systems 
%  \item<2-> situates simulations in the brain\rq s modal system (perceptual system)
%  \end{itemize}
\end{document}
